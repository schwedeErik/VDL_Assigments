\documentclass{article}
\usepackage{graphicx} % Required for inserting images
\usepackage{float}
\usepackage{amsmath}
\usepackage{bm,amssymb}
\usepackage{enumerate}
\author{Abdalla Arafa, 428044\\Ehsan Attar, 427214\\Sai Leela Poduru, 428272\\Prateek Rathod, 428396\\Shruti Shrivastava, 428364\\Erik Schwede, 428240}


\title{Exercise 2 - Convolutions and Loss Functions}
\date{November 2023}

\begin{document}

\maketitle
\section*{2.1. Convolutions}

\subsection*{i)}
\begin{table}[h]
  \centering
  \begin{tabular}{|c|c|c|c|c|}
    \hline
    -8 & -5 & -10 & 11 & 18 \\
    \hline
    -19 & -5 & -2 & 0 & 21 \\
    \hline
    -23 & 4 & -3 & -14 & 24 \\
    \hline
    -12 & 7 & -17 & -9 & 29 \\
    \hline
    -3 & 1 & -19 & 6 & 22 \\
    \hline
  \end{tabular}
\end{table}

\hspace{\fill}

\begin{table}[h]
  \centering
  \begin{tabular}{|c|c|c|c|c|}
    \hline
    -10 & -17 & -18 & -17 & -16 \\
    \hline
    -1 & -5 & 4 & 0 & -6 \\
    \hline
    -9 & 6 & 5 & -3 & -6 \\
    \hline
    8 & 13 & 3 & 5 & 13 \\
    \hline
    13 & 11 & 13 & 23 & 22 \\
    \hline
  \end{tabular}
\end{table}

\hspace{\fill}

\begin{table}[h]
  \centering
  \begin{tabular}{|c|c|c|c|c|}
    \hline
    33 & 13 & -20 & -18 & 3 \\
    \hline
    5 & 4 & -6 & 22 & -16 \\
    \hline
    -2 & -23 & 15 & -16 & -4 \\
    \hline
    -13 & 15 & 4 & 4 & -18 \\
    \hline
    -13 & 8 & -12 & -19 & 12 \\
    \hline
  \end{tabular}
\end{table}

\subsection*{ii)}
These filters are commonly used for edge detection in image processing: 
\begin{itemize}
    \item Prewitt Filter: Emphasizes both horizontal and vertical edges in an image.
    \item Laplacian Filter: Emphasizes regions of rapid intensity change.
    \item Sobel Filter: Emphasizes vertical edges in an image.
\end{itemize}

\subsection*{iii)}
In CNNs, "valid" padding involves no additional padding, resulting in a smaller output, while "same" padding maintains input dimensions by adding padding as needed, yielding an output size equal to the input.

\subsection*{iv)}
CNNs are preferred over MLPs for image data due to their ability to capture spatial hierarchies efficiently. The use of convolutional layers allows them to recognize local patterns and complex visual features, while weight sharing provides translation invariance. CNNs naturally form feature hierarchies, making them adept at image recognition tasks by exploiting spatial correlations. These architectural features make CNNs more effective for handling the inherent characteristics of image data compared to traditional MLPs.

\subsection*{v)}
\[ O = \frac{{H - k + 2p}}{{s}} + 1 \]

\subsection*{vi)}
\[ \text{Trainable parameters} = N \times (k \times k \times C + 1) \]
where:
\( N \) is the number of filters,
\( k \) is the size of each filter (assuming square filters, \( k \times k \)),
\( C \) is the number of channels in the input, and
\( 1 \) accounts for the bias term associated with each filter.
\newpage
\section*{2.2 Loss Functions and Optimization}
\begin{enumerate}[i)]
    \item Derivative of softmax
    \begin{alignat}{2}
        \hat{y_i}=\frac{e^{z_i}}{\sum_{k=1}^n e^{z_k}}
        \\
    \end{alignat}
    Derivative for condition i = j  using quotient rule:
    \begin{alignat}{2}
        \frac{\partial{\hat{y}_i}}{\partial{z}_i} &= \frac{e^{z_i}\sum e^{z_k} - e^{z_i} . e^{z_i}}{(\sum_k e^{z_k})^2}
        \\
        &= \frac{e^{z_i} (1 - e^{z_i})}{(\sum_k e^{z_k})^2}
        \\
        &= \hat y_i (1-\hat y_i)
    \end{alignat}
    Derivative for condition i != j  using quotient rule:
    \begin{alignat}{2}
        \frac{\partial{\hat{y}_i}}{\partial{z}_j} &= \frac{0- e^{z_i}.e^{z_j}}{(\sum_k e^{z_k})^2}
        \\
        &= -\hat{y}_i.\hat{y}_j
    \end{alignat}
    \item Derivative of cross-entropy loss function defined as
    \begin{alignat}{1}
        L(y, \hat{y}) = - \sum_{k=1}^N y_k \log\hat{y}
    \end{alignat}
    \begin{alignat}{2}
        \frac{\partial L}{\partial \hat{y_i}} = - \sum_i y_i . \frac{1}{\hat{y_i}}
    \end{alignat}
    Using chain rule:
    \begin{alignat}{2}
        \frac{\partial L}{\partial z_i} &= \frac{\partial L}{\partial \hat{y_i}} . \frac{\partial \hat{y_i}}{\partial z_i}
    \end{alignat}
    \begin{alignat}{10}
        \frac{\partial L}{\partial z_i} &= -  \sum_{i\not=j} y_i. \frac{1}{\hat{y_i}} \frac{\partial \hat{y_i}}{\partial z_i} - y_i.\frac{1}{\hat{y_i}}\frac{\partial \hat{y_i}}{\partial z_i}
        \\
        &= - \sum_{i\not=j} y_i.\frac{1}{\hat{y_i}} . (-\hat{y_i}\hat{y_j}) - y_i. \frac{1}{\hat{y_i}} . \hat{y_i} (1 - \hat{y_i})
        \\
        &= \sum_{i\not=j} y_i . \hat{y_j} + y_i \hat{y_i} - y_i
        \\
        &= \sum_i y_i . \hat{y_j} - y_i
        \\
        &= \hat{y_j} - y_i
    \end{alignat}
\end{enumerate}
\newpage
\section*{2.3 Loss Functions with Regularization}
Given,
\\
\[\hat{y} = Xw\]

\[\L(y,\hat{y_i}) = \frac{1}{n} * \sum_{i=1}^n ((\hat{y_i} - y_i)^{2}) +  \sum_{i=1}^d w_i^{2}\]

\[\hspace{0.5cm}  = \frac{1}{n} * \sum_{i=1}^n ((\hat{y_i} - y_i)^{2}) + {||w||}_{2}^{2} \]
Now, Let’s find derivative of L w.r.t w and set it to 0 to get minimum value of w 

\[\frac {\partial L}{\partial w} = 0\]

\[\implies \frac{1}{n} * \sum_{i=1}^n 2(\hat{y_i} - {y_i})({X_i}) + 2\lambda w = 0 \]

\[\implies \frac{1}{n} * X^{T} (\hat{y}-y) + \lambda w = 0\]

\[\implies w = \frac{1}{\lambda n} * X^{T}(y-\hat{y})\]

\[\therefore Closed \hspace{0.1cm} form\hspace{0.1cm} for\hspace{0.1cm} vector\hspace{0.1cm} w\hspace{0.1cm} that\hspace{0.1cm} minimizes\hspace{0.1cm} L\hspace{0.1cm} is\hspace{0.1cm} \]
\[w = \frac{1}{\lambda n} * X^{T}(y-Xw)\]
\\

\end{document}
